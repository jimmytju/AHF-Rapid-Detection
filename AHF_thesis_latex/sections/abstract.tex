\begin{abstract}
    \noindent Background and Objectives: Acute Heart Failure (AHF) results in over 26 million hospital admissions worldwide annually, posing a significant global health concern. Currently, AHF diagnosis relies on biochemical markers and echocardiography, which take more than 20 minutes. Auscultation, a quick and non-invasive clinical practice, is used alongside the gold standard. To address the need for rapid clinical AHF diagnosis, this paper presents a model for feature extraction and diagnosis using short heart sound signals.\\
    \noindent Methods: Discrete wavelet transform was applied for heart sound denoising, and Mel Frequency Cepstrum Coefficients were used for feature extraction. A new DenseHF-Net is proposed for diagnosing heart failure. Additionally, a feature fusion method is introduced for multi-region fusion auscultation, including mitral, aortic, and pulmonic valves, along with an ensemble method for long auscultation in the mitral valve region. \\
    \noindent Results: An auscultation dataset containing 2999 recordings, each with rich annotations, has been established. The proposed wavelet denoising algorithm achieved a signal-to-noise ratio of 7.8 dB. For multi-region fusion auscultation using DenseHF-Net, the average accuracy reached 99.25\%. For mitral valve ensemble auscultation, the average accuracy was 92.60\%.\\
    \noindent Conclusions: The proposed method enables rapid auscultation for AHF, providing accurate results based on a 3-second auscultation recording. While multi-region fusion auscultation achieves high accuracy, mitral valve ensemble auscultation offers a good balance between efficiency and accuracy. This research has the potential to be applied in cardiac auscultation tools for mobile phones, cloud-based systems, or electronic gloves. Data and codes are accessible on \href{https://github.com/qiuzhaoyu/AHF-Rapid-Diagnosis}{https://github.com/qiuzhaoyu/AHF-Rapid-Diagnosis}.\\
    \end{abstract}
    