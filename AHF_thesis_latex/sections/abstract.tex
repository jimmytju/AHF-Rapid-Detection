\begin{abstract}
\noindent $Background and objectives$: Acute Heart Failure (AHF) leads to over 26 million hospital admissions worldwide annually and is now a major global health concern. Currently, AHF diagnosis relies on biochemical markers and echocardiography, which takes more than 20 minutes. Auscultation, a quick and non-invasive clinical practice, is used alongside the gold standard. Recognizing the need for rapid clinical AHF diagnosis, this paper presents a model for feature extraction and diagnosis using short heart sound signals.\\

\noindent $Methods$: In this paper, discrete wavelet transform is applied for heart sound denoising, and the Mel Frequency Cepstrum Coefficient is applied for feature extraction. A new DenseHF-Net is proposed for the diagnosis of heart failure. A feature fusion method is proposed for multi-region fusion auscultation, including mitral, aortic, and pulmonic valves. An ensemble method is proposed for long auscultation in the mitral valve region.\\

\noindent $Results$: An auscultation dataset containing 2999 recordings has been established, each with rich annotations. A proposed wavelet denoising algorithm achieves a signal-to-noise ratio of 7.8 dB. For multi-region fusion auscultation, using DenseHF-Net, the average accuracy is 99.35\%. For mitral valve ensemble auscultation, using DenseHF-Net, the average accuracy is 94.41\%.\\

\noindent $Conclusions$: The above method enables rapid auscultation of AHF, providing accurate results based on a 3-second auscultation recording. Multi-region fusion auscultation achieves good auscultation accuracy, but mitral valve ensemble auscultation provides a good balance between efficiency and accuracy. The above research has the potential to be used for cardiac auscultation that can be used on mobile phones, cloud, or electronic gloves.\\
\end{abstract}