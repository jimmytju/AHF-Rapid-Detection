\section{Introduction}\label{sec:introduction}
\subsection{Background}
Cardiovascular disease (CVD), including heart failure (HF) and stroke, has become the leading cause of death and disability worldwide \cite{virani2020heart,roth2020global,mensah2019global,boorsma2020congestion}. The upward trend in the age-standardized rate of CVD is occurring in almost all non-high-income countries \cite{roth2020global}. Heart failure is an end-stage in the development of CVD characterized by dysfunction of the contractile/stretching function of the heart. Acute heart failure is a leading cause of emergency hospital admissions,  particularly in the elderly population \cite{sinnenberg2020acute,arrigo2020acute}. Acute heart failure (AHF) accounts for more than 26 million hospital admissions each year, with a mortality rate of 20-30\%  \cite{chapman2019clinical}. Emergency procedures for AHF include ambulance response, door-to-balloon, emergency, and ward transfer. Shortening the first two procedures can significantly reduce patient mortality and morbidity \cite{victor2012door,fan2021effects}. Taking myocardial infarction as an example, which is one of the causes of AHF: the control group reduced ambulance response time by 15.3 minutes and door-to-balloon time by 36 minutes, decreased the length of hospital stay by 6.3 days, lowered the mortality rate by 12.33\%, and reduced the rehospitalization rate by 19.69\% \cite{fan2021effects}. Therefore, shortening the door-to-balloon time in the emergency department for patients with AHF, and even completing the basic disease diagnosis during the ambulance response, can significantly improve the survival rate of patients. However, the traditional process of AHF diagnosis needs to be optimized. While some rapid methods, such as auscultation, are part of clinical examinations, their effectiveness needs further quantitative evaluation.

Traditional AHF diagnosis relies heavily on clinical biochemical trials, and there is a lack of convenient and accurate diagnostic methods, especially for 20\% of patients with no history of chronic heart failure (CHF). According to the European Society of Cardiology's AHF First Aid Guidelines, the diagnostic criteria for AHF include clinical evaluation, electrocardiogram (ECG), chest x-ray and imaging techniques, laboratory tests, and echocardiography \cite{nieminen2005task}. The diagnosis of AHF is usually based on the history and physical examination as well as physical tests to assist in the examining of ECG, echocardiography, and biochemical tests of brain natriuretic peptide (BNP) and NT-proBNP. At present, according to ten mainstream NT-proBNP and BNP biochemical detection equipment on the market, the waiting time for BNP results is 9-16min, and the waiting time for NT-proBNP results is 11-21min \cite{lewis2020bnp}. The ejection fraction obtained by echocardiography is also an important indicator in the preoperative evaluation of AHF \cite{menon2022echocardiography}, which takes about 20-30 minutes for a single examination, including five minutes for equipment location.

Auscultation, as part of the clinical evaluation, is an effective means of diagnosing heart failure and is recommended by the European Society of Cardiology as class-1 \cite{nieminen2005task}. Heart sounds are produced by the mechanical motion of the heart's dynamic system. They are the sum of various mechanical vibrations caused by the blood movement within the cardiovascular system. Considering the mechanism by which heart sounds are produced, they contain a wealth of information regarding the physiology of the cardiovascular system and are regarded as a visual description of the contractile and stretch function of the heart \cite{johnston2007third,wynne2001clinical,boorsma2020congestion}. Normal heart sounds consist of 4 tones, called first heart sound (S1), second heart sound (S2), third heart sound (S3), and fourth heart sound (S4) in the order during the cardiac cycle. Mitral valve, aortic valve, and pulmonary valve are the three most commonly used auscultation areas. Different from ECG, heart sounds are a manifest of the ventricle's ability to pump blood, which can reflect pathological information better than ECG in a single cardiac cycle. In general, abnormal heart sounds have more background noise, greater S1 amplitude fluctuations and more high-frequency details than normal heart sounds. 
\subsection{Related works}
The current research on digital auscultation technology encompasses aspects such as datasets, signal processing, and diagnostic models.

In terms of datasets, the most commonly utilized dataset is the one established by PhysioNet for the Heart Sound Classification Challenge in 2016 \cite{clifford2016classification}, including 665 abnormal heart sounds and 2575 normal heart sounds. Yaseen et al. \cite{son2018classification} provided a five-classification dataset of Aortic Stenosis (AS), Mitral Regurgitation (MR), Mitral Stenosis (MS), Mitral Valve Prolapse (MVP) and Normal (N), with 200 cases of each type. Nonetheless, for the development of algorithms aimed at diagnosing a particular medical condition, the current dataset may exhibit limitations in terms of annotation richness. Therefore, to develop rapid diagnostic models for AHF, additional efforts are needed to refine the data and standardize inclusion criteria.

In terms of pre-processing and feature extraction for heart sounds, in Vepa's research \cite{vepa2009classification}, heart sound signals were processed based on Short-Time Fourier Transform (STFT) and Discrete Wavelet Transform (DWT). Wu et al. \cite{wu2010hidden} extracted the MFCC components of heart sound signals based on the hidden Markov model (HMM) model. Existing techniques, such as feature extraction based on MFCC, have been highly successful and widely utilized in the processing of heart sounds. However, current heart sound processing techniques are primarily designed for long-duration signals, with some even exceeding 60 seconds, rendering them unsuitable for AHF rapid diagnosis.

In terms of diagnostic models, Rubin \cite{rubin2016classifying} used a convolutional neural network (CNN) for heart sound signal classification by time-frequency characteristics. Arora et al. \cite{arora2021transfer} performed transfer learning of heart sound signals based on VIZ, MobileNet, Xception, VGG, ResNet, DenseNet and Inception. Scholars such as Li and Shuvo \cite{li2021lightweight,shuvo2021cardioxnet} have developed end-to-end lightweight neural networks for clinical mobile devices. The established models are too large in size for rapid diagnosis of AHF. Therefore, we have developed lightweight models and introduced different auscultation strategies for various clinical scenarios.

To address the challenges of signal processing and diagnostic model in AHF auscultation, our main contributions are:

\begin{itemize}
\item  An auscultation dataset has been established, containing 2999  recordings from heart failure patients, aiming at addressing the challenges in heart failure auscultation. This dataset encompasses comprehensive information, including diagnostic results, collection area annotations, medical history records, and annotations related to BNP and NT-proBNP. Additionally, this dataset has undergone a rigorous ethical review process and is publicly accessible on \href{https://github.com/jimmytju/AHF-Rapid-Diagnosis/Database}{https://github.com/jimmytju/AHF-Rapid-Diagnosis/Database}.
\item A wavelet denoising algorithm and a lightweight DenseHF-Net have been developed for short-duration auscultation signals, aiming at rapid diagnosis of acute heart failure. The denoising algorithm proposed in this paper has improved the average signal-to-noise ratio to 7.8 dB. DenseHF-Net has only 3.82M parameters and is easily ported to low-computation scenarios such as mobile terminals. 
\item We have introduced two auscultation strategies: multi-region fusion auscultation and mitral valve auscultation. Multi-region fusion auscultation is designed for scenarios where long-time auscultation is possible, such as monitoring situations. It utilizes the three most crucial regions, namely, the mitral valve region, aortic valve region, and pulmonic valve region. On the other hand, mitral valve auscultation is specifically designed for rapid diagnosis in cases of heart failure emergencies, focusing solely on the mitral valve auscultation region. It completes the diagnosis within 15 seconds with an accuracy of 94.41\%, representing only a 4.94\% reduction compared to the multi-region fusion strategy.
\end{itemize}

% The remainder of this paper is arranged as follows:
% Paragraph.\ref{Materials and Methods} describes data denoising, feature extraction and models building. Paragraph.\ref{Results} describes the results of Paragraph.\ref{Materials and Methods}. Paragraph.\ref{Discussion} discusses the experimental results and compares them with other research. Paragraph.\ref{Conclusion} summarises the results and shortcomings of this paper, and suggests future research.